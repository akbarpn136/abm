\chapter{Pendahuluan}

\maketitle% This tag is required to print author and address in the output

\lettrine[loversize=1,lraise=0.3,lines=2]{A}{gen} merupakan individu atau objek yang memiliki ciri khas dan tindakan serta tujuan untuk melakukan suatu perhitungan tertentu secara otomatis dalam suatu lingkungan. Agent Based Modeling (ABM) merupakan suatu fenomena interaksi antar agen atau agen dengan lingkungan dalam suatu lingkungan yang dapat dibuatkan ke dalam model komputasi. Lingkungan merupakan suatu bentangan di mana agen-agen dapat berinteraksi, dipetakan, memiliki hubungan, dan digambarkan ke dalam bentuk data. Melalui pendekatan ABM, dapat memudahkan dalam melakukan eksplorasi, visualisasi dan analisis fenomena alam yang terjadi.

\section{This is First Level Heading}

\lipsum[1-2]

\subsection{This is Second Level Heading}

\lipsum[3]

For example, multiple citations from the \index{bibliography} of this \index{article}:
citet: \citet{CR7,CR8}, citep: \citep{CR9,CR6}.
As you see in Table~\ref{tab1-2}, the citations are \index{automatically!hyperlinked!citations} to their
reference in the bibliography. Refer Figure~\ref{fig1-1} for details:

\begin{figure}
\includegraphics{01.eps}
\caption{Figure Caption. Figure Caption.
Figure Caption. Figure Caption. Figure Caption. 
Figure Caption. Figure Caption.
\label{fig1-1}}
\end{figure}

For example, multiple citations from the \index{bibliography} of this \index{article}:
citet: \citet{CR9,CR6}, citep: \citep{CR9,CR6}.
As you see in Table~\ref{tab1-1}, the citations are \index{main entry with turn over lines main entry with turn over lines!subentry with turn over lines subentry with turn over lines!subsubentry with turn over lines subsubentry with turn over lines} to their
reference in the bibliography (Equation~\ref{eq1-1}).
\begin{equation}
\mathcal{L}\quad \mathbf{\mathcal{L}} = i \bar{\psi} \gamma^\mu D_\mu \psi - 
\frac{1}{4} F_{\mu\nu}^a F^{a\mu\nu} - m \bar{\psi} \psi\label{eq1-1}
\end{equation}

\subsubsection{This is Third Level Heading}

The manifestation of solar activity (flares, bursts, and others) occurs over the whole Sun, and most of radio astronomy observations are made from the Earth's surface, whereas a significant part of solar radio events (those from the far side of the Sun) is not available for terrestrial observers.
\begin{equation}
\mathcal{L}\quad \mathbf{\mathcal{L}} = i \bar{\psi} \gamma^\mu D_\mu \psi - 
\frac{1}{4} F_{\mu\nu}^a F^{a\mu\nu} - m \bar{\psi} \psi\label{eq1-2}
\end{equation}

\paragraph{This is Fourth Level Heading}

\lipsum[5]

\begin{table}
\caption{Enter table caption here.\label{tab1-1}}{%
\begin{tabular}{@{}cccc@{}}
\toprule
Tap     &Relative   &Relative   &Relative mean\\
number  &power (dB) &delay (ns) &power (dB)\\
\midrule
3 &0$-9.0$  &68,900\footnotemark[1] &$-12.8$\\
4 &$-10.0$ &12,900\footnotemark[2] &$-10.0$\\
5 &$-15.0$ &17,100 &$-25.2$\\
\botrule
\end{tabular}}{\footnotetext[]{Source: Example for table source text.}
\footnotetext[1]{Example for a first table footnote. Example for a first table footnote. Example for a first table footnote. Example for a first table footnote.}
\footnotetext[2]{Example for a second table footnote.}}
\end{table}

\subparagraph{This is Fifth Level Heading}

\lipsum[6]

The manifestation of solar activity (flares, bursts, and others) occurs over the whole Sun, and most of radio astronomy observations are made from the Earth's surface, whereas a significant part of solar radio events (those from the far side of the Sun) is not available for terrestrial observers (Equation~\ref{eq1-2}).

\begin{table}
\caption{Enter table caption here.\label{tab1-2}}{%
\begin{tabular}{@{}cccc@{}}
\toprule
Tap     &Relative   &Relative   &Relative mean\\
number  &power (dB) &delay (ns) &power (dB)\\
\midrule
3 &0$-9.0$  &68,900\footnotemark[1] &$-12.8$\\
4 &$-10.0$ &12,900\footnotemark[2] &$-10.0$\\
5 &$-15.0$ &17,100 &$-25.2$\\
\botrule
\end{tabular}}{\footnotetext[]{Source: Example for table source text.}
\footnotetext[1]{Example for a first table footnote. Example for a first table footnote. Example for a first table footnote. Example for a first table footnote.}
\footnotetext[2]{Example for a second table footnote.}}
\end{table}

\backmatter

\bibliographystyle{plainnat}
\bibliography{wiley}%

