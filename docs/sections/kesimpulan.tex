\section{Kesimpulan}

Penggunaan agent based model (ABM) sangat mampu dalam memodelkan masalah sosial. Dengan membuat model berdasarkan interaksi dari para pelaku yang terlibat di dalamnya,  emergence  dapat termodelkan dengan baik tanpa melalui proses matematika yang rumit.

Peristiwa pemilih bayaran marak terjadi pada musim pemilihan. Hal tersebut kadang memberikan pengaruh lebih besar terhadap pemilih dibandingkan dengan pengaruh dari kualitas kandidatnya sendiri.  Simulasi yang dilakukan menggunakan NetLogo dan Mesa menunjukkan hasil

\begin{enumerate}
\item Pemberian uang meningkatkan tingkat keterpilihan, walaupun kandidat mempunyai kualitas yang tidak bagus (dibawah rata-rata)

\item Dalam kelompok pemilih yang relatif homogen (sosial-demografis), pemberian uang mempunyai pengaruh yang lebih kecil dibandingkan kelompok pemilih yang heterogenya

\item Pemberian uang dalam nominal yang lebih kecil, tetapi lebih luas penerimanya, akan menghasilkan suara yang lebih besar dibandingkan nominal yang lebih besar

\item Untuk meningkatkan kualitas pemilihan adalah dengan mencegah money-politics, meningkatkan kualitas kandidat, dan meningkatkan independensi pilihan pemilih
\end{enumerate}

Dengan demikian, money politik efektif dipengaruhi oleh seberapa luas uang tersebut disebarkan  bukan dari nominal uang yang ditawarkan. Meski demikian, money politik tidak menjamin hasil dari pemilihan karena ada faktor kualitas kandidat yang tinggi dan independensi dari pemilih yang masing-masing memiliki kecenderungan untuk membuat pilihan yang paling memberikan keuntungan secara pribadi.
