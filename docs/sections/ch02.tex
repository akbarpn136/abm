\section{Studi Pemilihan Pemimpin di ABM}

\lettrine{M}{etode} ABM sangat ampuh digunakan dalam memodelkan permasalahan sehari-hari karena dapat memodelkan masalaah tanpa melibatkan banyak proses matematika yang rumit \cite{crooks2018agent}. Pemilihan pemim-pin menjadi salah satu contoh masalah yang dapat dimodelkan di ABM. Munculnya sistem pemerintahan yang demokrasi membawa semakin banyak pihak yang terlibat dalam proses pembuatan keputusan dan memperbanyak variabel dalam menghitung peluang para calon untuk dapat memenangkan pemilihan \cite{feddersen2004rational}, terutama perilaku setiap orang dan interaksi didalamnya sangat berpengaruh terhadap hasil dari pemilihan. Banyak penelitian dilakukan untuk masalah pemilihan (voting) ini \cite{feddersen2004rational,saaty1989group}. Dalam skala kecil, masalah voting bisa hanya permasalahan membagi sekelompok murid ke dalam dua tim yaitu  merah dan biru, tetapi dalam masalah yang lebih besar dapat dikaitkan pada pemilihan presiden atau pemilihan pemimpin serikat negara. Kasus pemilihan pemimpin ini dapat diselesaikan menggunakan ABM dimana para pelaku pemilih dipandang sebagai agen-agen yang berperan didalamnya. Background dari agen dan cara berinteraksi dapat dijadikan sebagai dasar  dari aturan interaksi dalam membangun model dari pemilihan \cite{kazil2020utilizing}.

Banyak hal yang mempengaruhi setiap individu dalam melakukan pemilihan atau bahkan memilih untuk tidak memilih. Dalam artikel \cite{olson2012logic} terdapat konsep rasionalitas instrumental, yang menurutnya individu rasional membuat pilihan yang mereka yakini akan membawa hasil yang paling mereka sukai, Olson berpendapat bahwa ada sedikit insentif rasional bagi individu untuk berkontribusi pada produksi barang publik (atau umum), mengingat mereka (para calon) yang mengeluarkan biaya, para pemilih akan tetap memperoleh benefit terlepas dari ikut berkontribusi (suara) atau tidak \cite{savigny_2014}. Pandangan ini meningkatkan jumlah pemilih yang memilih untuk tidak memilih (golput) pada acara pemilihan.

Pada tahun 2019, di Indonesia melakukan pemilihan umum baik legislative maupun presiden. Menurut tim Lembaga Survey Indonesia (LSI), mereka menyatakan bahwa jumlah golput untuk pemilihan presiden di tahun 2019 yaitu sekitar 19.24\% yang telah menurun dibandingkat Pilpresa sebelumnya di tahun 2014 yang mencapai 30.42\% \cite{bbc_2019}. Presentase tersebut sangatlah tinggi sehingga banyak usaha yang dilakukan untuk meningkatkan minat pemilih untuk ikut berpartisipasi melalui berbagai cara, baik itu melalui hadiah dan promo bagi yang telah melakukan kewajibannya untuk memilih hingga ancaman bila tidak berpartisipasi.

Melakukan pemilihan sebenarnya adalah reaksi rasional dari setiap individu \cite{feddersen2004rational}. Salah satu hal yang banyak dilakukan adalah dengan menempatkan aktor-aktor strategis yang dapat mengarahkan pemikiran secara rasional dari individu untuk melakukan pemilihan. Dalam pemilihan pendahuluan, ada bukti bahwa pemilih mengkondisikan pilihan suara mereka pada kelayakan kandidat \cite{abramson1992sophisticated}. Dalam sebuah studi komprehensif, penulis \cite{cox1997making} menunjukkan bahwa pola pemungutan suara dan hasil pemilu secara umum konsisten dengan pola perilaku yang diprediksikan oleh model voting strategis, misalnya di bawah aturan pluralitas (di mana kandidat dengan suara terbanyak memenangkan pemilihan).
